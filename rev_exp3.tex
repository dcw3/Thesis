\documentclass{article}
\usepackage[utf8]{inputenc}
\usepackage{amsmath}
\usepackage{bbm}
\usepackage{amssymb}
\DeclareMathOperator*{\argmax}{arg\,max}
\DeclareMathOperator*{\argmin}{arg\,min}
\DeclareMathOperator{\EX}{\mathbb{E}}
\DeclareMathOperator{\RE}{\mathbb{RE}}
\newcommand{\ita}{\textit}
\newcommand{\eps}{\epsilon}
\newcommand{\R}{\mathbb{R}}
\newcommand{\p}{\mathbb{P}}

\title{Adversarial Bandit and EXP3 Review}
\author{dcw3}
\date{November 2018}

\begin{document}

\maketitle

\section{Introduction}

The adversarial bandit setting was introduced by Auer, Cesa-Bianchi, Freund, and Schapire. In the standard bandit setting, the payout distributions of the 

\end{document}